% Copyright (C) 2023 Alexandru Sima (312CA) %
\documentclass{article}

\usepackage[romanian]{babel}
\usepackage{fancyhdr}
\usepackage{fontawesome5}
\usepackage{graphicx}
\usepackage[hidelinks]{hyperref}
\usepackage[utf8]{inputenc}
\usepackage{listings}
\usepackage[a4paper, margin=0.9in]{geometry}


\begin{document}

\pagestyle{fancy}
\begin{titlepage}
      \begin{center}
            Universitatea POLITEHNICA din București \\
            Facultatea de Automatică și Calculatoare \\[5px]
            \includegraphics[width=1.2cm]{upb.jpg}
            \includegraphics[width=1.2cm]{acs.jpg}

            \vfill
            {\huge\bf SHA-256}
            \vfill
            Alexandru Sima (312CA) \\
            Iarina-Ioana Popa (312CA) \\
            \small\today
      \end{center}
\end{titlepage}

\fancyhead[L]{PCLP3}
\fancyhead[R]{Alexandru Sima, Iarina-Ioana Popa (312CA)}
\urlstyle{style=normal}
\tableofcontents
\newpage

\section{Descriere}
Proiectul constă în scrierea unui program care calculează funcția de hash
SHA-256 a unor fișiere/a unui input de la tastatură (funcționalitate
asemănătoare cu a utilitarului sha256sum).

Proiectul este hostat pe \faIcon{github}
\url{https://github.com/alx-sima/sha256}.

\section{Echipă}
\begin{itemize}
      \item Alexandru Sima
            \begin{itemize}
                  \item Versionare proiect
                  \item Implementare algoritm
                  \item Modularizarea proiectului
                  \item Testare cod
            \end{itemize}
      \item Iarina-Ioana Popa
            \begin{itemize}
                  \item Scriere Makefile
                  \item Implementare algoritm
                  \item Debug
                  \item Elaborare documentație
            \end{itemize}
\end{itemize}

\newpage
\section{De ce am ales acest proiect?}
\subsection{Motivație}
\begin{itemize}
      \item Algoritmii de hash sunt foarte importanți, deoarece ajută la
            verificarea integrității datelor, criptarea parolelor și crearea de
            semnături digitale.

      \item Criptarea parolelor este un mecanism de securitate esențial.

      \item Funcțiile de hash au proprietăți matematice interesante, precum
            imposibilitatea inversării lor fără brute-force și variația mică a
            inputului care duce la o variație mare a outputului.

      \item Având în vedere că am învățat importanța funcțiilor de hash semestrul
            trecut, ne-am gândit că ar fi interesant să vedem implementarea
            algoritmilor și să înțelegem matematica din spate.
\end{itemize}

\subsection{Concepte acoperite}
\begin{itemize}
      \item Funcții de hash
      \item Operații pe biți
      \item Macrodefiniții
      \item Proiecte compilate cu Makefile
      \item Versionarea proiectelor folosind VCS (git).
\end{itemize}

\newpage
\section{Algoritm}
\subsection{Ce este SHA-256?}
\begin{itemize}
      \item SHA-2 (Secure Hash Algorithm 2) este un set de algoritmi criptografici
            dezvoltați de NSA.

      \item Acesta a devenit standard în 2002.

      \item SHA-256 este o funcție din standardul SHA-2 care returnează un hash de
            256 de biți.
\end{itemize}

\subsection{De ce funcționează?}
\begin{itemize}
      \item Inițial, mesajul este expandat, ceea ce mărește complexitatea operații.
      \item Anumiți pași ai algoritmului implică rotații de biți și permutări,
            ceea ce amestecă biții și duce la o dispersie mai mare.
      \item În timpul procesului de hashing se pierd anumite informații din input,
            deoarece se aplică operații pe biți precum and sau shift.
      \item Rezultatul are o lungime fixă, deci mai multe surse pot avea același hash.
\end{itemize}

\subsection{Implementare}
\begin{itemize}
      \item  Se inițializează 8 variabile care reprezintă hashul (primii 32 de
            biți ai părților fracționare ale radicalilor primelor 8 numere prime)

      \item se preprocesează, adaugându-se un bit 1 la finalul acestuia și
            completându-se cu 0 și cu un întreg pe 64 de biți (reprezentând
            lungimea șirului) până la o lungime multiplu de 512 biți.

      \item Mesajul se prelucrează în chunkuri de câte 512 biți, astfel:
            \begin{itemize}
                  \item Se copiază mesajul în primele 16 variabile de lucru.
                  \item Se calculează încă 48 de variabile de lucru, aplicându-se
                        operații pe biți celor anterioare.
                  \item Se comprimă chunkul, aplicându-se operații pe biți în care
                        sunt implicate variabilele de lucru, hashul până la acest
                        moment și 64 de constante matematice (primii 32 de biți ai
                        părților fracționare ale rădăcinilor cubice ale primelor 64
                        de numere prime).
                  \item Se adaugă "hashul" chunkului comprimat la hashul total.
            \end{itemize}

      \item La final, se concatenează cele 8 variabile de hash.
\end{itemize}

\subsection[Pseudocod]{Pseudocod\footnote{\url{https://en.wikipedia.org/wiki/SHA-2}}}

\begin{lstlisting}[basicstyle=\small,breaklines=true, frame=single]
Note 1: All variables are 32 bit unsigned integers and addition is calculated modulo 232
Note 2: For each round, there is one round constant k[i] and one entry in the message schedule array w[i], 0 <= i <= 63
Note 3: The compression function uses 8 working variables, a through h
Note 4: Big-endian convention is used when expressing the constants in this pseudocode,
and when parsing message block data from bytes to words, for example,
the first word of the input message "abc" after padding is 0x61626380

Initialize hash values:
(first 32 bits of the fractional parts of the square roots of the first 8 primes 2..19):
h0 := 0x6a09e667
h1 := 0xbb67ae85
h2 := 0x3c6ef372
h3 := 0xa54ff53a
h4 := 0x510e527f
h5 := 0x9b05688c
h6 := 0x1f83d9ab
h7 := 0x5be0cd19

Initialize array of round constants:
(first 32 bits of the fractional parts of the cube roots of the first 64 primes 2..311):
k[0..63] :=
0x428a2f98, 0x71374491, 0xb5c0fbcf, 0xe9b5dba5, 0x3956c25b, 0x59f111f1, 0x923f82a4, 0xab1c5ed5,
0xd807aa98, 0x12835b01, 0x243185be, 0x550c7dc3, 0x72be5d74, 0x80deb1fe, 0x9bdc06a7, 0xc19bf174,
0xe49b69c1, 0xefbe4786, 0x0fc19dc6, 0x240ca1cc, 0x2de92c6f, 0x4a7484aa, 0x5cb0a9dc, 0x76f988da,
0x983e5152, 0xa831c66d, 0xb00327c8, 0xbf597fc7, 0xc6e00bf3, 0xd5a79147, 0x06ca6351, 0x14292967,
0x27b70a85, 0x2e1b2138, 0x4d2c6dfc, 0x53380d13, 0x650a7354, 0x766a0abb, 0x81c2c92e, 0x92722c85,
0xa2bfe8a1, 0xa81a664b, 0xc24b8b70, 0xc76c51a3, 0xd192e819, 0xd6990624, 0xf40e3585, 0x106aa070,
0x19a4c116, 0x1e376c08, 0x2748774c, 0x34b0bcb5, 0x391c0cb3, 0x4ed8aa4a, 0x5b9cca4f, 0x682e6ff3,
0x748f82ee, 0x78a5636f, 0x84c87814, 0x8cc70208, 0x90befffa, 0xa4506ceb, 0xbef9a3f7, 0xc67178f2

Pre-processing (Padding):
begin with the original message of length L bits
append a single '1' bit
append K '0' bits, where K is the minimum number >= 0 such that (L + 1 + K + 64) is a multiple of 512
append L as a 64-bit big-endian integer, making the total post-processed length a multiple of 512 bits
such that the bits in the message are: <original message of length L> 1 <K zeros> <L as 64 bit integer> , (the number of bits will be a multiple of 512)

Process the message in successive 512-bit chunks:
break message into 512-bit chunks
for each chunk
    create a 64-entry message schedule array w[0..63] of 32-bit words
    (The initial values in w[0..63] don't matter, so many implementations zero them here)
    copy chunk into first 16 words w[0..15] of the message schedule array

    Extend the first 16 words into the remaining 48 words w[16..63] of the message schedule array:
    for i from 16 to 63
        s0 := (w[i-15] rightrotate  7) xor (w[i-15] rightrotate 18) xor (w[i-15] rightshift  3)
        s1 := (w[i-2] rightrotate 17) xor (w[i-2] rightrotate 19) xor (w[i-2] rightshift 10)
        w[i] := w[i-16] + s0 + w[i-7] + s1

    Initialize working variables to current hash value:
    a := h0
    b := h1
    c := h2
    d := h3
    e := h4
    f := h5
    g := h6
    h := h7

    Compression function main loop:
    for i from 0 to 63
        S1 := (e rightrotate 6) xor (e rightrotate 11) xor (e rightrotate 25)
        ch := (e and f) xor ((not e) and g)
        temp1 := h + S1 + ch + k[i] + w[i]
        S0 := (a rightrotate 2) xor (a rightrotate 13) xor (a rightrotate 22)
        maj := (a and b) xor (a and c) xor (b and c)
        temp2 := S0 + maj

        h := g
        g := f
        f := e
        e := d + temp1
        d := c
        c := b
        b := a
        a := temp1 + temp2

    Add the compressed chunk to the current hash value:
    h0 := h0 + a
    h1 := h1 + b
    h2 := h2 + c
    h3 := h3 + d
    h4 := h4 + e
    h5 := h5 + f
    h6 := h6 + g
    h7 := h7 + h

Produce the final hash value (big-endian):
digest := hash := h0 append h1 append h2 append h3 append h4 append h5 append h6 append h7
\end{lstlisting}


\end{document}
